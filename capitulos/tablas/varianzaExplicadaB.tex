\begin{table}[h]
    \caption {Varianza total explicada del bienestar}
	\label{tab:varianzaExplicadaB}
	\setlength\extrarowheight{5pt}
	
	\begin{tabular}{p{1.5cm} p{1.9cm} p{1.7cm} p{1.7cm} p{1.9cm} p{1.9cm} p{1.7cm}}
	\toprule
	Comp.	& \multicolumn{3}{c}{Valores propios iniciales} & \multicolumn{3}{c}{Rotación de las sumas de los cuadrados} \\
	\midrule
		& Total	& \% de varianza	& \% acumulado	& Total	& \% de varianza 	& \% acumulado \\
	\midrule
	1  & 3,886 & 48,574 & 	48,574 & 3,886 & 48,574 & 48,574 \\
	2  & 1,010 & 12,628 & 	61,202 &  &  &  \\
	3  & 0,901 & 11,265 &	72,467 &  &  &  \\
	4  & 0,652 & 8,146 &	80,613 &  &  &  \\
	5  & 0,612 & 7,650 &	88,264 &  &  &  \\
	6  & 0,396 & 4,951 &	93,215 &  &  &  \\
	7  & 0,297 & 3,719 &	96,933 &  &  &  \\
	8  & 0,245 & 3,067 &	100,000 &  &  &  \\
	\bottomrule
	\end{tabular}
	
	\center
	\footnotesize
	Método de extracción: Análisis del componente principal.\\
	Fuente: elaboración propia
\end{table}
