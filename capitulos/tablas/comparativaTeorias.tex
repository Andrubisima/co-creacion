%\newgeometry{left=1cm}
\begin{landscape}
\begin{table}[h]
    \caption {Comparativa de las teorías asociadas a la co-creación de valor.}
	\label{tab:comparativaTeorias}
	\setlength\extrarowheight{5pt}
	
	\begin{tabular}{p{1.8cm} p{3.8cm} p{3.8cm} p{3.8cm} p{3.8cm} p{3.8cm} }
	\toprule
		                 & LDP                                                                      & LDS                                       & LS                                       & Servucción                                                                                    & Experiencias                                                          \\ \midrule
	El valor             & Valor en el intercambio.                                                 & Valor en el contexto.                     & Valor en uso a partir de la interacción. & La co-creación de valor se realiza entre la empresa y el cliente.                             & Valor en la experiencia.                                              \\
	El cliente           & Usa y destruye el valor que ha creado la empresa a través del consumo. & Co-creador de valor.                      & Creador y co-creador de valor.           & Es productor y consumidor.                                                                    & Creador de valor a través de co-creación de experiencias.             \\
	El rol de la empresa & Produce y distribuye valor.                                              & Realiza proposiciones de valor.           & Facilitadora o realizadora de valor.     & Cuida el soporte físico, los recursos humanos y la administración y organización.             & Facilitadora de entornos de experiencias o de plataformas engagement. \\
	El servicio          & Practicamente inexistente                                                & Da solución o aplicación de competencias. & Es la actividad.                         & Sin cliente no existe el servicio.                                                            & Experiencias personalizadas.                                          \\
	Los recursos         & Unidades de output y recursos operandos.                                 & Operantes y operandos.                    & Recursos y procesos de apoyo.            & Ventajas competitivas enfocadas hacia la oferta total, hacia el soporte de la oferta o ambas. & DART: diálogo, acceso, riesgo y beneficio y transparencia.            \\ \bottomrule
	\end{tabular}

	\center
	\footnotesize
	Fuente: Adaptado de Blasco, 2014. p. 61.
\end{table}
\end{landscape}
