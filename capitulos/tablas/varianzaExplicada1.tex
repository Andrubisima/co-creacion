\begin{table}[h]
    \caption {Varianza total explicada estudio 1}
	\label{tab:varianzaExplicada1}
	\setlength\extrarowheight{5pt}
	
	\begin{tabular}{p{1.5cm} p{1.9cm} p{1.7cm} p{1.7cm} p{1.9cm} p{1.9cm} p{1.7cm}}
	\toprule
	Comp.	& \multicolumn{3}{c}{Valores propios iniciales} & \multicolumn{3}{c}{Rotación de las sumas de los cuadrados} \\
	\midrule
		& Total	& \% de varianza	& \% acumulado	& Total	& \% de varianza 	& \% acumulado \\
	\midrule
	1  & 8,042 & 44,680 & 	44,680 & 5,059 & 28,106 & 28,106 \\
	2  & 2,285 & 12,697 & 	57,376 & 4,054 & 22,524 & 50,629 \\
	3  & 1,648 & 9,157 &	66,534 & 2,863 & 15,904 & 66,534 \\
	4  & 1,176 & 6,534 &	73,068 &  &  &  \\
	5  & 0,934 & 5,188 &	78,256 &  &  &  \\
	6  & 0,835 & 4,638 &	82,894 &  &  &  \\
	7  & 0,711 & 3,952 &	86,846 &  &  &  \\
	8  & 0,560 & 3,112 &	89,958 &  &  &  \\
	9  & 0,516 & 2,866 &	92,823 &  &  &  \\
	10 & 0,304 & 1,689 & 	94,513 &  &  &  \\
	11 & 0,259 & 1,437 & 	95,949 &  &  &  \\
	12 & 0,219 & 1,217 & 	97,166 &  &  &  \\
	13 & 0,185 & 1,028 & 	98,194 &  &  &  \\
	14 & 0,125 & 0,692 & 	98,886 &  &  &  \\
	15 & 0,110 & 0,611 & 	99,498 &  &  &  \\
	16 & 0,050 & 0,277 & 	99,775 &  &  &  \\
	17 & 0,025 & 0,138 & 	99,913 &  &  &  \\
	18 & 0,016 & 0,087 & 	100,000 &  &  &  \\
	\bottomrule
	\end{tabular}
	
	\center
	\footnotesize
	Método de extracción: Análisis del componente principal.\\
	Fuente: elaboración propia
\end{table}
