\begin{table}[h]
    \caption {Varianza total explicada de la satisfacción}
	\label{tab:varianzaExplicadaS}
	\setlength\extrarowheight{5pt}
	
	\begin{tabular}{p{1.5cm} p{1.9cm} p{1.7cm} p{1.7cm} p{1.9cm} p{1.9cm} p{1.7cm}}
	\toprule
	Comp.	& \multicolumn{3}{c}{Valores propios iniciales} & \multicolumn{3}{c}{Rotación de las sumas de los cuadrados} \\
	\midrule
		& Total	& \% de varianza	& \% acumulado	& Total	& \% de varianza 	& \% acumulado \\
	\midrule
	1  & 3,468 & 57,798 & 	57,798 & 3,468 & 57,798 & 57,798 \\
	2  & 0,837 & 13,945 & 	13,945 & &  &  \\
	3  & 0,611 & 10,181 &	10,181 & &  &  \\
	4  & 0,518 & 8,634 &	8,634 &  &  &  \\
	5  & 0,323 & 5,382 &	5,382 &  &  &  \\
	6  & 0,244 & 4,060 &	4,060 &  &  &  \\
	\bottomrule
	\end{tabular}
	
	\center
	\footnotesize
	Método de extracción: Análisis del componente principal.\\
	Fuente: elaboración propia
\end{table}
