\section{Tema a tratar, justificación y objetivos}

El presente trabajo tiene como objetivo arrojar luz sobre la co-creación de valor. El principal motivo para la orientación y desarrollo de este trabajo fin de grado en el sector hotelero, ha sido el deseo de conocer más en profundidad la importancia que adquieren los huéspedes para las organizaciones hoteleras.

El desarrollo de las nuevas tecnologías y la globalización ha permitido que prácticamente todos los clientes y empresas puedan acceder a la misma información al mismo tiempo. Por ello, las ventajas competitivas se encuentran en una continua evolución en la que el cliente tiene mucho que aportar a la empresa y viceversa. Como consecuencia, la co-creación de valor se ha convertido en una importante fuente de ventaja competitiva.

El objetivo principal que se pretende, es el de conocer la relación existente entre el bienestar, la satisfacción, la participación, el surgimiento de problemas y la fidelización del huésped con la co-creación de valor. Para ello, primero se ha realizado un análisis de la bibliografía más relevante y se han expuesto las diferentes teorías desarrolladas por los autores más importantes. Tras la realización de encuestas a los huéspedes y una entrevista con el director del hotel se ha contrastado si estos factores tienen tienen una relación directa con la co-creación de valor mediante un análisis factorial y un análisis ANOVA.

\section{Estructura y metodología}

En el marco teórico, se ha creído conveniente definir el término de valor añadido, ya que es fundamental para poder comprender el concepto de co-creación de valor. A continuación, se realiza una recopilación de diferentes definiciones dadas en la literatura del término co-creación de valor, para proponer una definición propia que recoge los conceptos que se estiman más importantes de este término. Además, se han expuesto las teorías más trascendentes que han desarrollado diversos autores sobre la co-creación de valor hasta nuestros días y se exponen las diferencias que existen entre cada una de ellas.

La parte empírica del trabajo es la correspondiente al capítulo \ref{section:parteEmpirica}. Se trata de conocer, a través del estudio estadístico, la relación de la co-creación de valor con el bienestar, la satisfacción, la participación, los problemas y la fidelización del huésped en el hotel seleccionado. Para ello, se ha analizado un hotel de la capital de Austria, Viena. Dicho estudio se ha llevado a cabo a través de los resultados de una encuesta orientada al cliente con preguntas relacionadas con diferentes aspectos de la co-creación de valor, la satisfacción, el bienestar, la participación, la fidelización del huésped y los problemas que pueden surgir a lo largo de la estancia. Con estos datos se ha realizado un análisis factorial y un contraste de hipótesis. Además, se ha tenido la oportunidad de realizar una entrevista al director del hotel para poder contrastar los resultados obtenidos en las encuestas con la visión por parte de la compañía hotelera. Estos resultados han sido tratados y analizados en el capítulo \ref{section:parteEmpirica}.

Para finalizar el trabajo, se han expuesto las principales conclusiones surgidas a lo largo del proyecto. Además, se han adjuntado anexos para cada capítulo en los que se amplía información y se detallan los cuestionarios utilizados en la parte empírica.
