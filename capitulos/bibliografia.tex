\bibliographystyle{plain}
\bibliography{biblio}

\begin{thebibliography}{11}
	
	\bibitem{}
		Aitken, R., Ballantyne, D., Osborne, P.,  \& Williams, J. (Eds.) (2006). Introduction to the special issue on the service-dominant logic of marketing: Insights from The Otago Forum. \emph{In Marketing Theory, 6}(3), 275-280.

	\bibitem{}
		Anderson, J. C., \& Narus, J. A. (2008). Business Marketing: Understand What Customers Value - HBR [Web log post]. Recuperado de  \url{https://hbr.org/1998/11/business-marketing-understand-what-customers-value}.

	\bibitem{}
		Bartl, M. (2009). \emph{Co-Creation in the Automobile Industry - The Audi Virtual Lab}. Recuperado de \url{michaelbartl.com}. 	

	\bibitem{}
		Bartl, M. (2009). \emph{Methods and Tools for Co-Creation and Open Innovation}. Recuperado de \url{michaelbartl.com}. 

	\bibitem{}
		Bartl, M. (2009). \emph{The Making-of Innovation, The Morphology of Co-Creation}. Recuperado de \url{http://www.michaelbartl.com/article/the-morphology-of-co-creation/}. 

		
	\bibitem{}
		Bartl, M. (2009). \emph{The Morphology of Co-Creation}. Recuperado de \url{www.michaelbartl.com/co-creation}. 

	\bibitem{}
		Bartl, M. \& Füller, J. (2007). \emph {User Design in Practice – The Audi Virtual Lab}. Presentado en World Conference on Mass Customization \& Personalization  (MCP), MIT, Cambridge, MA. 

	\bibitem{}
		Bartl, M. \& Füller, J. (2007). \emph {What Consumers Expect from Virtual Co-Creation}. Presentado en World Conference on Mass Customization \& Personalization  (MCP), MIT, Cambridge, MA. 

	\bibitem{}
		Bateson, J. E., \& London Business School. (1983). \emph{Perceived control and the service encounter}. London: London Business School. 
	
	\bibitem{}
		Binkhorst, E. (2006).  \emph{The co-creation tourism experience}. Recuperado de \url{http://www.esade.edu/cedit2006/pdfs2006/papers/esther_binkhorst_paper_esade_may_06.pdf}	 

	\bibitem{}
		Blasco, L. (2014). \emph{Los procesos de co-creación y el engagement del cliente: un análisis empírico en medios interactivos}. (Tesis doctoral, Universidad de Zaragoza, Zaragoza, España). 

	\bibitem{}
		Blasco, L., Hernández, B., \& Jiménez, J. (2012). \emph{Co-creation processes and engagement: an empirical approach}. Universidad de Zaragoza. 

	\bibitem{}
		Blodgett, J., Granbois, D., \& Walters, R. (1993). The effects of perceived justice on complainants' negative word of mouth behavior and repatronage intentions. (Artículo).  \emph{Journal of Retailing, 69}(4), 399.

	\bibitem{}
		Brandenburguer, A. M., \& Harbone, W. S. (1.996). Value-Based Business Strategy. \emph{Management Economics and Strategy}. Recuperado de \url{http://pages.stern.nyu.edu/~abranden/ValueBasedStrategy2.pdf}

	\bibitem{}
		Breidbach, C. F., Kolb, D. G., \& Srinivasan, A. (2013). Connectivity in service systems: Does technology-enablement impact the ability of a service system to co-create value?  \emph{Journal of Service Research, 16} (3), 428-441.

	\bibitem{}
		Cervantes, V. H. (2005). Interpretaciones del coeficiente Alpha de Cronbach.  \emph{Avances en Medición, 3}, 9-28.

	\bibitem{}
		Chen, T., Drennan, J., \& Andrews, L. (2012). Experience sharing. \emph{Journal of Marketing Management, 28}(13-14), 1535-1552.

	\bibitem{}
		Chevalier, J. \& Mayzlin, D. (2006). The Effect of Word of Mouth on Sales: Online Book Reviews.  \emph{Journal of Marketing Research, 48}, 345-354. 

	\bibitem{}
		Churchill, G. A. (1979). A paradigm for developing better measures of marketing construct. \emph{Journal of Marketing Research, 16}(1), 64-73.

	\bibitem{}
		Cook, Scott (2008). The Contribution Revolution: Letting Volunteers Build your Business. \emph{Harvard Business Review, 86} (octubre), 60-69. 	

	\bibitem{}
		Cuadras, C. M. (2014). \emph{Nuevos métodos de análisis multivariante} (Tesis doctoral, Universidad de Barcelona, Barcelona, España). 

	\bibitem{}
		Definición de Enterprise 2.0 - Forbes. (2010). Recuperado de \url{http://www.forbes.com/2010/01/19/twitter-facebook-web-technology-cio-network-enterprise.html}

	\bibitem{}
		Deighton, J.,  \& Kornfeld, L. (2010). \emph{United breaks guitars HBS Case No. 510–057: Harvard Business School Marketing Unit}.	

	\bibitem{}
		Demey, J. R., Adams, M., \& Freites, H. (1992, agosto 25). Uso del método de analisis de componentes principales para la caracterización de fincas agropecuarias. \url{http://sian.inia.gob.ve/repositorio/revistas_ci/Agronomia%20Tropical/at4403/Arti/demey_j.htm}

	\bibitem{}
		Diz-Comesaña, M.E. \& Rodríguez-López, N. (2011). La participación del cliente como co-creador de valor en la prestación del servicio.  \emph{INNOVAR, 21}(41), 159-168.

	\bibitem{}
		Dodds, W.B. (1991) In Search of Value: How Price and Store Name Information Influence Buyers’ Product Perceptions. \emph{Journal of Services Marketing 5}(3), 27-36.

	\bibitem{}
		Drucker, P. F. (1993). \emph{Management: Tasks, responsibilities, practices}. New York: HarperBusiness. 

	\bibitem{}
		Eaves, A. (2010). \emph{El valor de una solución completa y real para el cliente}. Ediciones Deusto.

	\bibitem{}
		Eiglier, P., Langeard, E., \& Mollá, D. A. (1989). \emph{Servucción: El marketing de servicios}. Madrid, España: McGraw-Hill. 

	\bibitem{}
		Ethical Leadership and Creating Value for Stakeholders (2004). En Robert A. Peterson y O.C. Ferrell (Eds.) Business Ethics: 82–97. M.E. Sharpe, Armonk, NY, London.

	\bibitem{}
		Everitt, B. S. \& Hothorn, T. (2011). \emph{An Introduction to Applied Multivariate Analysis with R}. New York: Springer-Verlag. ISBN 978-1-4419-9649-7.

	\bibitem{}
		Firat, A. F., \& Venkatesh, A. (1995). Liberatory Postmodernism and the Reenchantment of Consumption. \emph{Journal of Consumer Research}. Doi:10.1086/209448.

	\bibitem{}
		Fisk, R., Grove, S., Harris, L. C., Keeffe, D., Reynolds, K. L., Russell-Bennett, R., \& Et. al. (2010). Customers behaving badly: A state of the art review, research agenda and implications for practitioners. \emph{Journal of Services Marketing, 4}(6), 417-429.

	\bibitem{}
		Fisher, K., \& Statman, M. (2.002). Blowing Bubbles. \emph{The Journal of Psychology and Financial Markets, 3}.

	\bibitem{}
		Fitzsimmons, J. A. (1985) Consumer participation and productivity in service operactions Fitzsimmons, consumer participation and productivity in service operactions. \emph{Interfaces, 15} (3), 60-67. Recuperado de \url{http://pubsonline.informs.org/doi/abs/10.1287/inte.15.3.60?journalCode=inte}

	\bibitem{}
		Fodness, D., Pitegoff, B. E., \& Sautter, E. T. (1993). From customer to competitor: consumer cooption in the service sector. \emph{of Services Marketing, 7}(3), 18-25.  		

	\bibitem{}
		Forbes. (2010).  \emph{Enterprise 2.0}. url{http://www.forbes.com/2010/01/19/twitter-facebook-web-technology-cio-network-enterprise.html} 

	\bibitem{}
		Frances, E. (2013). \emph{The effects of co-creation and satisfaction on subjective well-being}. (Tesis doctoral), Faculty of the Virginia Polytechnic Institute, Virginia, Unites States.  

	\bibitem{}
		Gebauer, J., Füller, J., \& Pezzei, R. (2013). The dark and the bright side of co-creation: Triggers of member behavior in online innovation communities.  \emph{Journal of Business Research, 66}, 1516-1527.  Recuperado de \url{http://dx.doi.org/10.1016/j.jbusres.2012.09.013}.	

	\bibitem{}
		Gefen, D., Karahanna, E., \& Straub, D. W. (2003). Inexperience and experience with online stores: The importance of TAM and trust.  \emph{IEEE Transactions on Engineering Management, 50} (3).

	\bibitem{}
		García Montalvo, J., \& Raya Vilchez, J. M. (2.012). What is the right price of Spanish residential real estate? \emph{SEFO - Spanish Economic and Financial Outlook, 1}, 3.	
	
	\bibitem{}
		Godes, D., \& Mayzlin, D. (2004), Using Online Conversations to Study Word-of-Mouth Communication.  \emph{Marketing Science, 23} (Fall), 545-560. 

	\bibitem{}
		Gómez, M. (2005). \emph{El análisis factorial en investigación comercial}. Universidad autónoma de Madrid.  

	\bibitem{}
		Goodwin, C. (1988). I can do it myself: training the service consumer to contribute to service productivity. \emph{Journal of Services Marketing, 2} (4), 71-78.

	\bibitem{}
		Grönroos, C. (2006).  \emph{defining marketing: finding a new roadmap for marketing (6)}. Recuperado de \url{http://www.sagepub.com/clow/study/articles/PDFs/01_Gronroos.pdf} 	

	\bibitem{}
		Grönroos, C. (2006). What Can a Service Logic Offer Marketing Theory?, in Lusch, R.F. \& Vargo, S.L. (eds.)  \emph{The Service-Dominant Logic of Marketing: Dialog, Debate, and Directions}, 354-364. Armonk, NY: M.E. Sharpe. Recuperado de \url{http://www.researchgate.net/publication/215915799_Adopting_a_service_logic_for_marketing}

	\bibitem{}
		Grönroos, C., \& Voima, P. (2011). Critical service logic: Making sense of value creation y cocreation.\emph{Journal of the Academy of Marketing Science}. Doi: 10.1007/s11747-012-0308-3.
	\bibitem{}
		Gwinner, K.P., D.D. Gremler \& M.J. Bitner (1998): Relational Benefits in Services Industries: The Customer’s Perspective. \emph{Journal of the Academy of Marketing Science,26} (2), 101-114. Recuperado de \url{http://bit.ly/1d9a7ge}

	\bibitem{}
		Hair, J. F., Anderson, R. E., Tatham, R. L., \& Black, W. C. (1999). \emph{Multivariate data analysis}. New Jersey: Prentice Hall.

	\bibitem{}
		Hamel, G., Doz, Y. L., \& Prahalad, C. K. (1989). Collaborate with Your Competitors - and Win. 

	\bibitem{}
		Holbrook, B. (1994). Ethics and the Typology of Customer Value by N. Craig Smith. Recuperado de \url{http://www.acrwebsite.org/search/view-conference-proceedings.aspx?Id=7931}

	\bibitem{}
		Huang, M., \& Rust, R. T. (2013). IT-related service: A multidisciplinary perspective. \emph{Journal of Service Research, 16}(3), 251-258.

	\bibitem{}
		Hunt, D. M., Geiger-Oneto, S., \& Varca, P. E. (2012). Satisfaction in the context of customer co-production: a behavioural involvement perspective. \emph{Journal of Consumer Behaviour, 11}, 347-356.

	\bibitem{}
		IBM. (2015). IBM Downloading IBM SPSS Statistics 23 - United States.  Recuperado de \url{http://www-01.ibm.com/support/docview.wss?uid=swg24038592}

	\bibitem{}
		Información empresarial | Facebook Newsroom. (2014). Recuperado de \url{https://newsroom.fb.com/company-info/}

	\bibitem{}
		Johnston, R. (1989). The customer as employee.  \emph{International Journal of Operations and Production Management, 9}(5), 15-23.  

	\bibitem{}
		Jöreskog, K., \& Sörbom, D. (1993).  \emph{LISREL 8 structural equation modeling with the simplis comand language}. Chicago-Illinois: Scientific Software International.

	\bibitem{}
		Kotler, P. (2010). \emph{Marketing 3.0: From products to customers to the human}. New York: Wiley.

	\bibitem{}
		Kotler, P. (2014). Las tres orientaciones del marketing: Producto, Cliente, Persona.  Recuperado de \url{http://manuelgross.bligoo.com/content/view/1025608/Philip-Kotler-Las-tres-orientaciones-del-marketing-Producto-Cliente-Persona.html}

	\bibitem{}
		Kwon, K., \& Kim, C. (2012). How to design personalization in a context of customer retention: Who personalizes what y to what extent? \emph{Electronic Commerce Research y Applications, 11}(2), 101-106.

	\bibitem{}
		Lapierre, J. (2000). \emph{‘Customer-Perceived Value in Industrial Contexts’, The Journal of Business \& Industrial Marketing 15}(2-3), 122-140.

	\bibitem{}
		Lemos, S., Vallejo, G., \& Sandoval, M. (2002). Estructura factorial del Youth Self-Report (YSR). \emph{Psicothema, 14}(4), 816-822. ISSN: 0214 - 9915. 	

	\bibitem{}
		Likert, R (1932): A technique for the measurement of attitudes.\emph{Archives of Psychology, 140}(1), 44-53. 	

	\bibitem{}
		Lilian, G. (2004).  \emph{La servucción: Una herramienta para la gestión}. Paper presentado en el XXVII Congreso argentino de profesores universitarios de costes, Tandil, Buenos Aires.  

	\bibitem{}
		Lovelock, C. (2010).  \emph{Services marketing: People, technology, strategy} (7º ed.). Upper Saddle River: Pearson Education. 

	\bibitem{}
		Lovelock, C. H., \& Young, R. F. (1979). Look to consumers to increase productivity. \emph{Harvard Bussines Review, 57}, 168-178. 

	\bibitem{}
		Maram, L. (n.d.). Qué es el marketing de atracción; 3 ejemplos | luisMARAM. Recuperado de \url{http://www.luismaram.com/2014/04/18/que-es-el-marketing-de-atraccion-3-ejemplos/}

	\bibitem{}
		Marketing Science Institute (2008), \emph{Research Priorities: 2008-2010}. Cambridge, MA: Marketing Science Institute.  

	\bibitem{}
		McDaniel, C. D., \& Gates, R. H. (1999). \emph{Investigación de mercados contemporánea}. México, D.F: International Thomson. 

	\bibitem{}
		Megias, J. (2012, enero 17). Herramientas: El mapa de empatía (entendiendo al cliente) | Startups, Estrategia y Modelos de negocio [Web log post].  Recuperado de \url{http://javiermegias.com/blog/2012/01/herramientas-el-mapa-de-empata-entendiendo-al-cliente/}

	\bibitem{}
		Méndez, C., \& Sepúlveda, M. A. (2012). Introducción al análisis factorial exploratorio.  \emph{Revista Colombiana de Psiquiatría, 41} (1), 197-207.

	\bibitem{}
		Meuter, M. L., Bitner, M. J., Ostrom, A. L., \& Brown, S. W. (2005). Choosing among alternative service delivery modes: An investigation of customer trial of self-service technologies. \emph{ournal of Marketing, 69}(3), 61-83.

	\bibitem{}
		Mitchell, J. (2003). \emph{Abrace a sus clientes: El método probado para personalizar las ventas y lograr resultados sorprendente}. Bogotá: Grupo Editorial Norma. 

	\bibitem{}
		Montoya, O. (2007). \emph{Aplicación del análisis factorial a la investigación de mercados. Casos de estudio} (35). Scientia et Technica XIII. 

	\bibitem{}
		Mora, H. (2010).  \emph{Breve guía de procedimientos para explorar la validez y confiabilidad de cuestionarios. aplicaciones con SPSS}.

	\bibitem{}
		Munuera, A. J., \& Rodríguez, E. A. (2007).  \emph{Estrategias de marketing: un enfoque basado en el proceso de dirección}. Madrid.

	\bibitem{}
		O’Hern, M. S. \& Rindfleisch, A. (2009), Customer Co-Creation: A Typology and Research Agenda, \emph{Review of Marketing Research, 6}. Naresh K. Malholtra, ed. Armonk, NY: M.E. Sharpe, 84-106.

	\bibitem{}
		Oliden, P. E., \& Zumbo, B. D. (2008). Coeficientes de fiabilidad para escalas de respuesta categórica ordenada.  \emph{Psicothema, 20}(4), 896-901. ISSN 0214 - 9915.

	\bibitem{}
		Oliver, R. L. (1980). A cognitive model of the antecedents and consequences of satisfaction decisions. (Article). \emph{Journal of Marketing Research (JMR), 17}(4), 460-469.

	\bibitem{}
		ÖRF. (2015). Medienforschung ORF [medienforschung.orf.at]. Recuperado de \url{http://mediaresearch.orf.at/c_fernsehen/console/console.htm?y=4&z=1}

	\bibitem{}
		Parasuraman, A., Zeithaml, V. A., Berry, L. L., \& Marketing Science Institute. (1986). \emph{Servqual, a multiple-item scale for measuring customer perceptions of service quality. }. Cambridge, MA: Marketing Science Institute. 

	\bibitem{}
		Pardo, A. \& Ruiz, M. A. (2002). \emph{SPSS: Guía para el análisis de datos}. Madrid: McGraw-Hill. 

	\bibitem{}
		Parkin, M., \& Sánchez, C. M. (2004). \emph{Economía}. México. Pearson Educación.

	\bibitem{}
		Payne, A. F., Storbacka, K., \& Frow, P. (2008). Managing the co-creation of value. \emph{Journal of the Academy of Marketing Science, 36}(1), 89-96.

	\bibitem{}
		Peña, D. (2002). \emph{Análisis de datos multivariantes}. McGraw Hill. Recuperado de \url{https://es.scribd.com/doc/132365997/Pena-Daniel-Analisis-de-Datos-Multivariantes-2002-pdf}

	\bibitem{}
		Pita, S., \& Pértega, S. (2001). Relación entre variables cuantitativas.  \emph{Cad atención primaria, 4}, 141-144.

	\bibitem{}
		\emph{Plataformas crowdsourcing}. (2011). Recuperado de \url{http://ucrowdsourcing.com/crowdsourcing.html. } 

	\bibitem{}
		Prahalad, C. K., \& Ramaswamy, V. (2004). Co-creation experiences: the next practice in value creation. \emph{Journal of Interactive Marketing, 18}(3), 5-14.

	\bibitem{}
		Prahalad, C. K., \& Ramaswamy, V. (2004). Co-creating unique value with customers. \emph{Strategy \& Leadership, 32}(3), 4-9.

	\bibitem{}
		Prahalad, C. K., \& Krishnan, M. S. (2008).  \emph{The new age of innovation: Driving cocreated value through global networks}. New York: McGraw-Hill Professional. 

	\bibitem{}
		Porter, M. (1996). \emph{Competitive Strategy}. ISBN: 0684841487.

	\bibitem{}
		Ramaswamy, V., \& Gouillart, F. (2010). \emph{The power of co-creation}. New York, US: Free Press.


	\bibitem{}
		Reise, S. P., Waller, N. G., \& Comrey, A. L. (2000). Factor analysis and Scale Revision. \emph{Psychological Assessment, 12}. (3), 287-297. \url{10.1037//1040-3590.12.3.287}.

	\bibitem{}
		 Rejonsaari, K. (2013). \emph{Co-creating health}. (Tesis doctoral). Aalto University, Hensinki, Finlandia.

	\bibitem{}
		ReVelle, J. B., Moran, J. W., \& Cox, C. A. (1998).\emph{The QFD handbook}. New York: Wiley.

	\bibitem{}
		Rolán-Alvarez, E. (n.d.). \emph{El ANOVA de un factor} [PowerPoint]. Recuperado de \url{http://rolan.webs.uvigo.es/statistics_course/Tema4.pdf}

	\bibitem{}
		Ryan, R. M. \& Deci, E. L. (2000). \emph{Intrinsic and extrinsic motivations: Classic definitions and new directions}. Contemporary Educational Psychology.

	\bibitem{}
		Sawhney, M. (2006). Going beyond the product: Defining, designing, and delivering customer solutions. \emph{In R. F. Lusch, \& S. L. Vargo (Eds.)}, The service-dominant logic of marketing: Dialog, debate, and direction ( 365–380). Armonk, NY: M.E. Sharpe.

	\bibitem{}
		Sawhney, M. (2013): Musings on Technology, Marketing and Life.  Recuperado de \url{http://www.mohansawhney.com/#!articles/cuma}

	\bibitem{}
		Sheth, J.N., Newman, B.I. \& Gross, B.L. (1991). Why We Buy What We Buy: A Theory of Consumption Values, \emph{Journal of Business Research 22}(2), 159–70.

	\bibitem{}
		Sheth, J. N., \& Parvatiyar, A. (1995). Relationship marketing in consumer markets: Antecedents y consequences.  \emph{Journal of the Academy of Marketing Science, 23}(4), 255-271.

	\bibitem{}
		Sirgy, J. (2013).  \emph{Psychology of Quality of Life Research}. Dordrecht: Springer Verlag. 

	\bibitem{}
		Smant, D. (2.001). Mississippi Bubble Dave Smant. Recuperado de \url{https://sites.google.com/site/davesmant/monetary-economics/famous-first-bubbles/mississippi-bubble}

	\bibitem{}
		Tejada, M. C. (2015, Marzo 29). El nuevo diario. Recuperado de \url{www.elnuevodiario.com.do/mobile/article.aspx?id=17531}

	\bibitem{}
		Thompson, C. J., Rindfleisch, A., \& Arsel, Z. (2006). Emotional branding and the strategic value of the doppelgänger brand image. \emph{Journal of Marketing, 70}(1), 20-64. 

	\bibitem{}
		Thorndike, R. L. (1989). Psicometría aplicada. Mexico: Limusa.
		Ulaga, W., \& Eggert, A. (2003).  \emph{Developing a Estandar Scale of Relationship Value in Business Markets: Development of a Measurement Scale}Paper presented at 17th Annual IMP Conference Proceedings. Recuperado de \url{http://www.impgroup.org/uploads/papers/271.pdf)}

	\bibitem{}
		Varadarajan, R., Srinivasan, R., Vadakkepatt, G. G., Yadav, M. S., Pavlou, P. A., Krishnamurthy, S., et al. (2010). Interactive technologies y retailing strategy: A review, conceptual framework y future research directions. \emph{Journal of Interactive Marketing,24}(2), 96-110.

	\bibitem{}
		Vargo, S. L., \& Lusch, R. F. (2008). Evolving to a new dominant logic for marketing. \emph{Journal of Marketing, 68}, 1-17.

	\bibitem{}
		Vargo, S. L., \& Lusch, R. F. (2008). Service-dominant logic: Continuing the evolution.  \emph{Journal of the Academy of Marketing Science} (in press). 

	\bibitem{}
		Vega-Vázquez, M., Revilla-Camacho, M. A., \& Cossío-Silva, F. J. (2013). The value co-creation process as a determinant of customer satisfaction.  \emph{Esmerald Group Publishing Limited, 51}(10), 1945-1953. Doi 10.1108/MD-04-2013-0227.

	\bibitem{}
		Viscarri, J. (2011, octubre 7). Modelo de creación de valor para el cliente. Recuperado de \url{http://upcommons.upc.edu/e-prints/bitstream/2117/16640/1/Viscarri_modelo_creacion_valor_cliente.pdf} 

	

\end{thebibliography}
