A modo de conclusión, se va a proceder a exponer brevemente las principales características de las diferentes teorías desarrolladas acerca de la co-creación de valor que se han podido deducir al concluir este trabajo.

Como se ha podido ver, se diferencian tres ramas para explicar la co-creación de valor. En primer lugar, la lógica dominante del producto que desembocó finalmente en la lógica dominante del servicio donde Vargo y Lusch fundamentan que la co-creación de valor surge de la integración de recursos. En segundo lugar, las teorías que sostienen que a través de la interacción entre la empresa y el cliente se crea valor, rompiendo así con los procesos de producción de productos para desarrollar uno propio para los servicios. Esas teorías son desarrolladas por Eigler y Langeard con la teoría de la servucción; y la escuela nórdica de Grönroos que desarrolla la lógica del servicio. Y por último, Prahalad y Ramaswamy que defienden las relaciones personalizadas entre la empresa y el cliente para que éste último pueda interaccionar con el entorno empresarial y cree experiencias de co-creación de valor tanto para la empresa como para sí mismo. Estas teorías, se han tenido en cuenta a la hora de formular las hipótesis de la parte empírica.

Se ha podido comprobar la importancia de la co-creación de valor para las empresas en cuanto a los recursos que pueden proponer a los clientes y a los valores que las empresas quieren adoptar para inculcarlos a sus empleados y que éstos se los transmitan a los clientes a la hora de interaccionar. Todos estos aspectos, hacen ver que existen otras formas de generación de valor para la empresa que están adquiriendo gran importancia. Al igual que se ha podido contemplar que la co-creación de valor también es importante para el cliente. Por ejemplo, en el hecho de volver a consumir ese bien o servicio, a la hora de publicitar a la empresa si se encuentra satisfecho o por el contrario ejercer una publicidad negativa a través del boca a boca. Por lo tanto, la co-creación de valor puede ser un arma de doble filo.

En vista de la variedad de definiciones que la literatura aporta sobre la co-creación de valor y la diversidad de ramas que han surgido para poder dar explicación a este fenómeno, en este trabajo, se ha planteado una definición propia, que es la siguiente:

\emph{La co-creación de valor es la interacción entre el cliente y la empresa (Grönroos, 2012) en diferentes momentos como el diseño, la producción y el consumo o disfrute del bien o servicio (Seth, Sisodia Y Sharma, 2000) integrando los recursos de ambos (Vargo y Lusch, 2004) y creando de este modo un entorno experiencial en el que los clientes puedan tener un diálogo activo para poder construir así sus propias experiencias y percepciones personalizadas (Prahalad y Ramaswamy, 2004).}

En cuanto a la parte empírica de este trabajo, se puede afirmar que la satisfacción, la participación y la fidelización de los huéspedes tienen relación con la co-creación de valor; y por el contrario, el bienestar y el surgimiento de problemas en el hotel no tienen relación con ella. Se trata de unas soluciones esperadas en su mayoría en cuanto a los contrastes de hipótesis se refiere a excepción del bienestar, ya que podría darse el caso de que un cliente que no perciba bienestar participe en los procesos de co-creación. Se está de acuerdo que el surgimiento de los problemas durante la estancia del huésped en el hotel va a propiciar que el cliente no participe activamente en estos procesos de co-creación. Y por último,  un huésped que está satisfecho, que participa, que forma parte del programa de fidelidad de una forma activa y que se encuentra satisfecho con este programa de fidelización, es razonable que forme parte de los procesos co-creativos de valor.

Lo que también se puede afirmar es que la co-creación de valor es el presente y el futuro empresarial. En un mundo globalizado en el que tanto empresa como cliente tienen acceso a la información prácticamente al mismo tiempo, lo que marca la diferencia son las ventajas competitivas. Para ello, las empresas tienen que ser conscientes de que permitir y fomentar la participación de los clientes en los procesos de co-creación e interactuar con ellos les permite co-crear valor y por lo tanto adquirir una ventaja competitiva sostenible.
