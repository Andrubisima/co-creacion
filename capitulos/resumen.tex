\section*{Resumen}

En el presente trabajo se tratará de profundizar en el concepto de co-creación de valor, analizando las diferentes definiciones existentes, y explicando las teorías más relevantes propuestas por los autores. Todo ello, con el principal objetivo de conocer qué relación tiene la co-creación de valor en un hotel con respecto a la satisfacción, el bienestar, la participación, la fidelización de los clientes y los problemas que les pueden surgir a estos. Para ello, se ha llevado a cabo una serie de encuestas a los huéspedes de un hotel en Viena y se ha realizado una entrevista al director de dicho hotel para obtener ambas visiones acerca de la co-creación de valor. Con todo ello, se han obtenido unos resultados que se reflejan en la parte empírica del trabajo y se han extraído unas conclusiones finales.

\section*{Palabras clave}

Co-creación de valor; Lógica dominante del producto; Lógica dominante del servicio; Experiencias; Interacción; Análisis factorial; ANOVA.

\section*{Abstract}

In this paper will seek to study the concept of co-creation of value, comparing various existing definitions and explaining the most important theories proposed by the authors. The main objective of this study is to know what relationship there is between the co-creation value in a hotel, and the level of satisfaction, participation, well-being and loyalty of its customers, as well as understanding any problems that might arise. To do so, a series of guest surveys at a hotel in Vienna were held, and the director of the hotel was interviewed, to obtain information and opinions about this relationship. The results of this empirical study are shown and conclusions are drawn. 


\section*{Key words}

Co-creation value; Goods dominant logic; Service dominant logic; Experiences; Interaction; Factor analysis; ANOVA.


